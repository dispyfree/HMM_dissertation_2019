\subsection{Working in Log Space}

A commonly encountered problem in numerical computation is underflow and overflow. Computers allocate a limited amount of memory to store numbers; hence, numbers must neither grow to large or too small, nor can all numbers be represented to an arbitrary accuracy. To which accuracy numbers are stored depends on a number of factors, most notably the programming language and system architecture. For most every-day applications however, those restrictions are negligible. 

In the following, we will 
\begin{itemize}
	\item show that the computations our algorithms conduct are indeed affected by numerical underflow
	\item present a mathematical approach to address this issue
	\item discuss the limitations of this approach
\end{itemize}


\subsubsection*{Numerical Underflow}
